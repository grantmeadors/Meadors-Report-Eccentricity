\documentclass{article}
\begin{document}
\title{Eccentricity's effect on TwoSpect parameter estimation \\ 
LIGO-T1500594}
\author{Grant David Meadors}
\date{\today}
%\affiliation{AEI Hannover/Golm}

\maketitle

\section{Introduction}
Reviewing TwoSpect, we must understand how eccentric orbits could affect TwoSpect.

Begin with Leaci \& Prix [PRD 91 (2015) 102003], who provide an equation (B1) for the instantaneous relative frequency offset of gravitational waves from a neutron star in a binary system, when that offset is caused by orbital motion.
The offset is expresed by a model of phase $\phi$, gravitational wave frequency $f$, parametrized by orbital angular frequency $\Omega = 2 \pi/P$ for period $P$, as well as argument of periapsis $\omega$ and eccentricity $e$.
Eccentric anomaly $E$ is a complicated function of time $t$.
Let the coefficient $a_\mathrm{p} = (a \sin i)/c$, for inclination $i$ and semi-major axis $a$ (the projected semi-major axis is $a \sin i$). 
Also let $(\delta f) / f$ denote the instantaneous relative frequency offset; then,

\begin{equation}
\left|\frac{\delta f}{f}\right| = \left| \frac{d \phi/dt}{2 \pi f} - 1 \right|,
\label{irfo_def}
\end{equation}
\begin{equation}
\left| \frac{d\phi/dt}{2\pi f} -1 \right| = a_\mathrm{p} \Omega \left|\frac{\sqrt{1-e^2} \cos E \cos \omega - \sin E \sin \omega}{1-e \cos E} \right|.
\label{irfo}
\end{equation}

In TwoSpect papers, such as the methods paper by Goetz \& Riles [CQG 28 (2011) 215006], the observed frequency modulation depth, $\Delta f_\mathrm{obs}$, is discussed.
This modulation depth is a one of the two search dimensions in the Scorpius X-1 directed search (along with $f$) and can be defined,

\begin{eqnarray}
\Delta f_\mathrm{obs} &= a_\mathrm{p} \Omega f \nonumber \\
    &= \frac{2\pi a \sin i}{cP} f.
\label{moddepth_def}
\end{eqnarray}

In the $e = 0$ case, $\delta f = \Delta f_\mathrm{obs} \cos E$; in this zero-eccentricity limit, $\Delta f_\mathrm{obs}$ is well-defined. 
Then the following are both equal to $\Delta f_\mathrm{obs}$,

\begin{enumerate}
\item frequency-modulation depth (modulation amplitude)
\item half of the peak-to-peak modulation width 
\end{enumerate}

TwoSpect's $R$-statistic templates are weighted heavily toward the modulation-depth extrema, because those are stationary points.
Therefore, the latter, \textit{peak-to-peak}, definition guides TwoSpect's parameter estimation.
Simulation software, namely MakeFakeData, takes $a \sin i$ as input.
Yet TwoSpect templates implicitly assume $e=0$, as they include no eccentricity parameter.
TwoSpect's inference from the highest-statistic templates to the modulation depth $\Delta f_\mathrm{obs}$ (and $a \sin i$) may therefore be skewed, if $e \neq 0$. 
 
The inference from the highest-statistic templates to frequency $f$ may likewise be skewed.
In the $e=0$ case, $f$ is well-defined, equal to both the following quantities:

\begin{enumerate}
\item intrinsic gravitational-wave frequency
\item the mean (half the sum) of the max and min modulation frequencies
\end{enumerate}

TwoSpect, being dominated by the modulation-depth extrema, is guided by the latter definition.
Again, MakeFakeData takes $f$ as input.
This means that $f$ estimation may also be skewed, if $e \neq 0$.

Indeed, this is what injection studies have shown. 
Can Equation~\ref{irfo} account for the observed skew as an effect of the instantaneous relative frequency offset?
In the rest of this note we show that this appears to be the case.

\section{Derivation}

\textit{Note: different sign conventions are in use for the argument of periapsis, which affects the sign of the observed offsets. 
Absolute values circumvent this difficulty.}

In the following, $f$ and $\Delta f_\mathrm{obs}$ (defined by Equation~\ref{moddepth_def}) are fixed quantities for a given injection; they correspond to astrophysical parameters.
We are studying the parameter estimation error in those quantities.
Call the \textit{apparent frequency offset} (from eccentricity effects) $\delta f_\mathrm{a}$.
The apparent modulation depth will be $\Delta f_\mathrm{a}$; subtracting the fixed $\Delta f_\mathrm{obs}$ from $\Delta f_\mathrm{a}$ yields the \textit{apparent modulation depth offset}, $\delta \Delta f_\mathrm{a}$.
It is $\delta f_\mathrm{a}$ and $\delta \Delta f_\mathrm{a}$ that generate parameter mis-estimates for TwoSpect.
Where $\max$ and $\min$ functions extremize over $E$,

\begin{equation}
\delta f_\mathrm{a} = \frac{1}{2} \left| \max(\delta f) + \min(\delta f)\right|,
\label{dfa}
\end{equation}
\begin{equation}
\delta \Delta f_\mathrm{a} = \frac{1}{2} \left| \max(\delta f) - \min(\delta f)\right| - \Delta f_\mathrm{obs}.
\label{ddfa}
\end{equation}

The extremal $\delta f_\mathrm{a}$ and $\delta \Delta f_\mathrm{a}$ depend on Equation \ref{irfo}: the form of the equation immediately shows that an extremum in $E$ and $\omega$ is reached when $E = 0$ and $\omega = n \pi$ for integer $n$.
Finding $\max(\delta f)$ and $\min(\delta f)$ for general $\omega$ requires solving for $E$,

\begin{eqnarray}
0 = &\frac{-e \sin E}{\left(1 - e \cos E \right)^{2}} \left(\sqrt{1-e^2} \cos E \cos \omega - \sin E \sin \omega \right) +\ldots \nonumber \\
    &\frac{-1}{\left(1-e \cos E \right)} \left(\sqrt{1-e^2} \sin E \cos \omega + \cos E \sin \omega \right),
\end{eqnarray}
\begin{equation}
0 = \sqrt{1-e^2} \sin E \cos \omega + (\cos E - e) \sin \omega.
\label{maxE}
\end{equation}

\noindent where in the second equality we assume $e\leq 1$ (the orbit is bound).
Solving this is easiest with a computer algebra system such as Wolfram Alpha, which verifies that $E = \pi n$ are extremal if $\sin \omega = 0$, and states that in other bound orbits the solutions are

\begin{eqnarray}
E_1 = 2 \pi n - 2 \tan^{-1} \left(\frac{\sqrt{1-e^2} \tan\left(\frac{\omega}{2}\right)}{1+e} \right),
\label{maxE1}
\end{eqnarray} 
\begin{eqnarray}
E_2 = 2 \pi n + 2 \tan^{-1} \left(\frac{\sqrt{1-e^2} \cot\left(\frac{\omega}{2}\right)}{1+e} \right).
\label{maxE2}
\end{eqnarray}

\noindent For low eccentricity, $e \ll 1$, $e\geq 0$, we can approximate these solutions as $E_1 = 2\pi n - \omega$, $E_2 = 2\pi n + \pi - \omega$.
These solutions do solve Equation~\ref{maxE} for $e=0$.
The MDC paper by Messenger et al [PRD 92 (2015) 023006] interprets Galloway et al [Astrophys J 781 (2014) 14] as a $3-\sigma$ upper limit on $e$ at 0.068, so this approximation seems useful.

Up to an overall sign ambiguity (inherited from Equation~\ref{irfo}, affecting Equations~\ref{dfa} and~\ref{ddfa}), we will calculate $\max(\delta f)$ using $E_1 = -\omega$ and $\min(\delta f)$ using $E_2 = \pi-\omega$:

\begin{equation}
\max(\delta f) = \Delta f_\mathrm{obs} \left(\frac{\sqrt{1-e^2} \cos^2 \omega + \sin^2 \omega}{1-e \cos (\omega)} \right),
\label{maxdeltaf}
\end{equation}
\begin{equation}
\min(\delta f) = -\Delta f_\mathrm{obs} \left(\frac{\sqrt{1-e^2} \cos^2\omega + \sin^2 \omega}{1+e \cos \omega} \right),
\label{mindeltaf}
\end{equation}

\noindent So inserting these into Equations~\ref{dfa} and ~\ref{ddfa} yields

\begin{equation}
\delta f_\mathrm{a} = \frac{1}{2} \left( \frac{\sqrt{1-e^2} \cos^2 \omega + \sin^2 \omega}{1-e^2 \cos^2 \omega} e \cos \omega \right) \Delta f_\mathrm{obs},
\label{solved_dfa}
\end{equation}
\begin{equation}
\delta \Delta f_\mathrm{a} = \frac{1}{2} \left( \frac{\sqrt{1-e^2} \cos^2 \omega + \sin^2 \omega}{1-e^2 \cos^2 \omega} \right)  \Delta f_\mathrm{obs} - \Delta f_\mathrm{obs}.
\label{solved_ddfa}
\end{equation}

Equations~\ref{solved_dfa} and~\ref{solved_ddfa} correctly predict the observed offsets, up to a sign (which is dependent on the argument of periapsis convention).
It is easy to verify that for $e=0.07$, they predict that an $a\sin i = 1.95$ light-s, $f=2039.95$ Hz signal will have an extremal case at $\omega=0$ with shifts of 25.7 mHz in $\delta f_\mathrm{a}$ and 0.9 mHz in $\delta \Delta f_\mathrm{a}$, which is what is seen in injections.

The equations are related, taking care when $e$ or $\omega$ equals $0$, by

\begin{equation}
\delta f_\mathrm{a} = \left(\delta \Delta f_\mathrm{a} + \Delta f_\mathrm{obs} \right) e \cos \omega,
\end{equation}
\begin{equation}
\delta \Delta f_\mathrm{a} = \frac{\delta f_\mathrm{a}}{e \cos \omega} - \Delta f_\mathrm{obs}.
\end{equation}

%And for low eccentricity cases, the following approximations can be useful:


%\begin{equation}
%\delta f_\mathrm{a} = \frac{e \Delta f_\mathrm{obs}}{1-e^2\cos^2\omega} \cos \omega
%\end{equation}
%\begin{equation}
%\delta \Delta f_\mathrm{a} = \frac{e^2 \Delta f_\mathrm{obs}}{1-e^2\cos^2\omega} \cos^2\omega.
%\end{equation}

\end{document}
